% Diese Datei sollte nicht verändert werden. Die Einstellungen zm Deckblatt
% finden sich in den "Voreinstellungen". Aufgaben werden in der Datei "Aufgaben"
% erstellt und editiert.
\firstpageheadrule
\newgeometry{left=1.5cm,
        right=1.5cm,
        top=4cm,
        bottom=1cm}
\coverheader
	{Lehrkraft: \lehrname\linebreak
		Fach: \ufach \linebreak
		\lstniveau}
	{\medskip\Large{\textbf{\LEKArt \ \KAnr}} \linebreak
		\normalsize{\kathema}}
	{{\fontsize{12}{16}{\selectfont Name:\enspace\makebox[4cm]{\hrulefill}\linebreak Klasse:\enspace\makebox[4cm]{\hrulefill}\linebreak Datum:\enspace\makebox[4cm]{\hrulefill}}}}
\begin{coverpages}
%############################# DECKBLATT ################
\vspace*{-0.7cm}\hrule \vspace*{0.5cm}
\begin{large}
\textbf{Bearbeitungshinweise}
\end{large}
\begin{itemize}
\item Bearbeitet die Aufgaben, die mit dem Symbol \hier \ gekennzeichnet sind, auf dem Aufgabenblatt.
\item Die Bearbeitungszeit beträgt \bearbzeit .
\item Lest die Aufgaben in Ruhe und ganz genau durch.
\item Lösungswege und Rechnungen müssen nachvollziehbar sein.
\item Antwortsatz nicht vergessen!
\item Zugelassene Hilfsmittel: \textbf{\zglhilf}
\end{itemize}
\vspace*{0.3cm}
\begin{center}
\begin{huge}
Viel Erfolg!
\end{huge}
\linebreak\includegraphics[scale=0.2]{smiley}
\linebreak
\hdashrule[0.1ex]{18.5cm}{0.5mm}{3mm 3pt} 
\end{center}
\begin{center}
\cellwidth{2.2em}
\newcommand{\bwrtzeilen}{1}
\INTEGERDIVISION{\totalquestions}{11}{\sola}{\solb}
\ADD{\sola}{1}{\numrows}
\multirowgradetable{%
  \numrows
}[questions]

\vspace*{0.7cm}
Diese Klassenarbeit besteht aus \numquestions\ Aufgaben. Insgesamt waren \numpoints\ Bewertungseinheiten (BE) zu erreichen. \smallskip

Du hast \underline{\hspace*{1.5cm}} BE erreicht. Das sind \underline{\hspace*{1.5cm}} Prozent. \medskip


\framebox(350,50){\textbf{Notenpunkte:} \underline{\hspace*{2cm}} \hspace*{1cm} \textbf{Note:} \underline{\hspace*{2cm}}}


\end{center}

\vfill

\newcommand{\tabledef}{G}
\textbf{Notenspiegel}
\vspace*{-0.6cm}
\ifthenelse{\equal{\ntable}{\tabledef}}{\gtable}{\etable}
\centering \vspace*{-1cm}
$\varnothing =$
\end{coverpages}
\restoregeometry